%/////////////////////////////////////////////////////////%
%//                        PREAMBLE                     //%
%/////////////////////////////////////////////////////////%
\documentclass[10pt]{article}
%%%%%%%%%%%%%%%%%%%%%%%
%       Packages
%%%%%%%%%%%%%%%%%%%%%%%
%% Fonts and Symbols
%% --------------------------
\usepackage{
    amsmath,            % math operators
    amssymb,            % math symbols
    textcomp,           % Copyright and Registered symbols
    pifont,             % Includes the pretty cirled numbers
    stmaryrd,			% Arrows
}
%% Graphics
%% --------------------
\usepackage{
    graphicx,            % allows insertion of images
    subfigure,           % allows subfigures (a), (b), etc.
    tikz,				% allows drawing shapes and whatnotS
}                
\graphicspath{ {graphics/} }    % (graphicx) relative path to graphics folder    
%% Tables
%% --------------------------
\usepackage{
    booktabs,            % better tables, discourages vertical rulings
    multicol,            % allow multi columns
    tabularx,            % variable-width columns
}
%% Layout Alteration
%% --------------------------
\usepackage{            
    geometry,            % change the margins for specific PAGES
    parskip,             % no indenting on paras with a line between paras
    fancyhdr,            % fancy headers and footers
}
\geometry{               % specify page size options for (geometry)
    letterpaper,         % paper size
    margin=1in,			% specified independently with hmargin vmargin
}    
%% Units
%% --------------------------
\usepackage{
    siunitx,             % has S (decimal align) column type
}
\sisetup{input-symbols = {()},  % do not treat "(" and ")" in any special way
    group-digits  = false, % no grouping of digits
    %         load-configurations = abbreviations,
    %         per-mode = symbol,
}
%% Misc
%% --------------------------
\usepackage{
    pdfpages,            % import pdfs into this document
    url,                % allows urls to be added to document
    amssymb,
    amsmath,
    listings,
}
%/////////////////////////////////////////////////////////%
%//                        BODY                         //%
%/////////////////////////////////////////////////////////%

\begin{document}
	%%%%%%%%%%%%%%%%%%%%%%%
	%       Title Section
	%%%%%%%%%%%%%%%%%%%%%%%
	\pagenumbering{gobble}        % turn off page numbering  
	\begin{center}
		\begin{tabularx}{\textwidth}{>{\raggedright}X>{\setlength\hsize{1\hsize}\centering}X>{\raggedleft}X}     
			CEng 355            &    {\huge Assignment 5 }            &    Jakob Roberts\tabularnewline
			&    {\small  }              		  &    v00484900\tabularnewline
		\end{tabularx}    
	\end{center}  
	%%%%%%%%%%%%%%%%%%%%%%%
	% 	  Body
	%%%%%%%%%%%%%%%%%%%%%%%
	\vspace{10mm}
	\renewcommand{\arraystretch}{1.6}
	\section*{1}
	\begin{itemize}
		\item []
		\begin{lstlisting}
for (p = 0; p < 2; p++) {	
	for (q = 0; q < 2; q++) {
		for (i = p*64; i < (p+1)*64; i++) {
			for (j = q*64; j < (q+1)*64; j++) {
				Y[i] = Y[i] + A[i][j]*X[j];
			}
		}
	}
}
		\end{lstlisting}
		@p = 0, Y[0:63] is computed in 2 steps.
		\begin{itemize}
			\item [1:]
			use A[0:63][0:63] with X[0:63] @ q=0.
			\item [2:]
			use A[0:63][64:127] with X[64:127] @ q=1.
		\end{itemize}
		@p = 1, Y[64:127] is computed in 2 steps. 
		\begin{itemize}
			\item [1:]
			use A[64:127][0:63] with X[0:63] @ q=0.
			\item [2:]
			use A[64:127][64:127] with X[64:127] @ q=1.
		\end{itemize}
		
		To store A (64x64), 64 x 64 x 4 = 16KB of space\\
		To store two 128x1 blocks, 2 x 128 x 4 = 1KB of space\\
		Therefore Cache size should be \textbf{17KB}\\
	\end{itemize}
	%%%%%%%%%%%%%%%%%%%%%%%
	%\pagebreak
	%%%%%%%%%%%%%%%%%%%%%%%
	\section*{2}
	\begin{itemize}
		\item []
		\textbf{\textit{NOTE: Assume int of size 32bit}}\\\\
		With N = 256, int size X[256][256] \(\Rightarrow\) 256 x 256 x 4 = 256KB.\\
		Each row requires \(256\times4=1KB\) of space.\\
		For the outer loop, per iteration on index i, there are 2 reads and 1 write per row.\\
		Each row then requires \(3\times256=768\ accesses \).\\\\
		With 4 x 1KB pages: 4 page faults per 4 rows.\\
		Fault Rate: \(4/(4\times768)=0.13\%\)\\
		With 1 x 4KB pages: 1 page faults per 4 rows.\\
		Fault Rate: \(1/(4\times768)=0.03\%\)\\
	\end{itemize}
	%%%%%%%%%%%%%%%%%%%%%%%
	\pagebreak
	%%%%%%%%%%%%%%%%%%%%%%%
	\section*{3}
	\begin{itemize}
		\item[]
		\textbf{\textit{NOTE: Assume float of size 32bit}}\\\\
		With N = 256, float size X[256][256] \(\Rightarrow\) 256 x 256 x 4 = 256KB.\\
		For first for loop (trace), 256 accesses are done as it computes 1 entry per row along the diagonal.\\
		For the second nested loops, it is reading every single element in every row.  Because it is doing a read and assign, it will need to access twice per element, therefore 512 times per row.\\
		Total accesses are 256 from the trace, and 512 per row. \(\Rightarrow\ 256+(256\times512)=131,328\ accesses\)\\\\
		With 4 x 1KB pages: 256/4=64 page faults from first loop, 256/4=64 page faults from second loops.\\
		Fault Rate: \((256+256)/131,328=0.3899\%\)\\
		With 1 x 4KB pages: 256 page faults from first loop, 256 page faults from second loops.\\
		Fault Rate: \((64+64)/131,328=0.0975\%\)
	\end{itemize}
\end{document}
