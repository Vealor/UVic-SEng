
\documentclass[landscape,dvips,headrule,footrule]{foils}
\usepackage{amsmath}
\usepackage{amssymb}

\setlength{\parindent}{0in}
%\usepackage{amsmath}
%\usepackage{amssymb}
%\usepackage{boxproof}
%\usepackage{prooftree}
%\usepackage{multicol}
%\usepackage{epic}
\usepackage[pdftex]{color}
%\usepackage{color}

\newcommand{\F}{\mathtt{F}}
\newcommand{\T}{\mathtt{T}}
\renewcommand{\P}{\mathcal{P}}
\newcommand{\M}{\mathcal{M}}
\renewcommand{\d}[1]{\textcolor{red}{{\em #1}}}
\newcommand{\m}[1]{\textcolor{blue}{$#1$}}
\newcommand{\g}[1]{\textcolor{magenta}{#1}}
\renewcommand{\r}[1]{\textcolor{red}{#1}}
\renewcommand{\b}[1]{\textcolor{blue}{#1}}
\newcommand{\ra}{\rightarrow}
\newcommand{\str}{\{0,1\}^*}
\newcommand{\N}{\mathbb{N}}
\newcommand{\rand}{\mathsf{rand}}
\newcommand{\lsb}{\mathsf{lsb}}
\newcommand{\rshift}{\mathsf{rshift}}
\newcommand{\fh}[1]{\foilhead[-0.5in]{#1}}




\begin{document}

\MyLogo{Cook-Levin Theorem}
\leftheader{CSC320}
%\rightheader{Spring 2003}

\fh{Cook-Levin Theorem}

{\bf Theorem} (Cook 1971, Levin) SAT (in CNF form) is NP-complete

We have already shown that SATISFIABILITY is in NP. So now
we need to show that for \d{any} language \m{L \in  NP},
\m{L} can be reduced to SATISFIABILITY

So for, we have only shown reducibilities of a \d{single}
problem to another. How can we handle all problems in NP?

Give a  \d{generic} reduction, based on nondeterministic TM's:
\begin{itemize}
\item For any \m{L \in  NP},
there must be a polynomial time nondeterministic TM \m{M} 
which accepts \m{L}.
\item We will use this fact to show that for any \m{L \in  NP},
there is a polynomial time reduction \m{f_L} such that for any \m{x},
\m{x \in L} if and only if \m{f_L(x)} is satisfiable.
\end{itemize}

\fh{Defining \m{f_L}}

Suppose \m{M = (Q,\Sigma, \Gamma, \delta,q_0,  q_{accept})}, and that \m{p}
is a polynomial which bounds the running time of \m{M}. Assume
that \m{p(n) \geq n}.

Suppose that \m{Q} is
numbered as follows: \m{q_0, q_1, \dots, q_w},
where \m{q_1=q_{accept}}.
and that \m{\Gamma} is numbered
\m{s_0,s_1, \dots s_v},
where \m{s_0=\sqcup}.

We will number the tape cells \m{\dots,-2,-1,0, 1,2,\dots}. Note that if the
running time of \m{M} is bounded by \m{p(n)} then we can never
move right or left from cell \m{0} more than \m{p(n)} times, and so we never need to consider
tape squares with a number whose absolute value is higher than \m{p(n)}.

We now show how to create \m{f_L(x)} for any instance \m{x} of \m{L}. Let
\m{n = |x|}. 

\fh{The Variables}

We first specify the set of variables.
\begin{center}
\begin{tabular}{c|c|c}
Variable & Range & Intended meaning\\ \hline
\m{y_{i,k}}   & \m{1 \le i \le p(n)}    & At time \m{i}, \m{M}\\
            & \m{0 \le k \le w}       & is in state \m{q_k}\\ \hline
\m{h_{i,j}}   & \m{1 \le i \le p(n)}    & At time \m{i}, tape\\
            & \m{-p(n) \le j \le p(n)}  & head is at cell \m{j}\\ \hline
\m{r_{i,j,k}} & \m{1 \le i \le p(n)}    & At time \m{i}, tape\\
            & \m{-p(n) \le j \le p(n)}  & cell \m{j} contains\\
            & \m{0 \le k \le v}       & symbol \m{s_k}
\end{tabular}
\end{center}
\fh{The Clause Groups}
The clauses come in six groups, each of which impose a constraint
on satisfying truth assignments which force a legal accepting computation.

\fh{The Clause Groups}
\begin{tabular}{cl}
Clause group & \multicolumn{1}{c}{Restriction}\\ \hline
\m{G_1} & At each time \m{i}, \m{M} is in exactly\\
      & one state\\
\m{G_2} & At each time \m{i}, the tape head is \\
      & on exactly one cell\\
\m{G_3} & At each time \m{i}, each tape cell \\
      & contains exactly one symbol\\
\m{G_4} & At time \m{1}, the computation is in its\\
      & initial configuration\\
\m{G_5} & By time \m{p(n)}, \m{M} has entered state \m{q_1}\\
\m{G_6} & For each time \m{i}, every configuration at\\
      & time \m{i+1} follows in one step from \\
      & the configuration at time \m{i},\\
      & according to \m{\delta}
\end{tabular}

\fh{Inside the Clause Groups}

We will now take a look inside some of the clause groups.
For \m{G_1} we need for every \m{i, ~~ 1 \le i \le p(n)} a clause 
{\color{blue}
\[
\{y_{i,0}, y_{i,1}, \dots, y_{i,w}\}
\]}
which says that we are in \d{ some state}, and also for every pair
\m{j,j'}, \m{1 \le j < j' \le w} we need a clause
{\color{blue}\[
\{\overline{y_{i,j}},\overline{y_{i,j'}}\}
\]}
which says that
we are not in {\em both} states \m{q_j} and \m{q_{j'}}.

Groups \m{G_2} and \m{G_3} are similar. Group \m{G_5} just contains
the single clause \m{\{y_{p(n),1}\}}, which says that \m{M} is in the accepting
state \m{q_1} at time \m{p(n)}.

\fh{Inside Group 4}

\m{G_4} is made up of the following clauses:

\m{\{y_{1,0}\}} - \m{M} starts in state \m{q_0}

\m{\{h_{1,0}\}} - \m{M} starts scanning cell \m{0} 

\m{\{ r_{1,-p(n),0} \} , \{r_{1,-p(n)+1,0}\}, \dots,  \{r_{1, -1, 0}\},}\\
\m{ \{r_{1,0,k_1}\}, \dots, \{r_{1,n-1,k_n}\},}\\
\m{\{r_{1,n,0}\}, \dots, \{r_{1,p(n)+1,0}\}}

-The initial tape is  \m{s_{k_1}\dots s_{k_n}} followed
by \m{\sqcup}'s,
where \m{x=s_{k_1}s_{k_2}\dots s_{k_n}} 

(NOTE: this last clause is the only one which depends on the
actual value of \m{x} (compare to the Sudoku encoding!))

\fh{Inside Group 6}

This is the most complicated. Basically we need to say that every
configuration at step \m{i+1} must follow in one step from a configuration
at step \m{i}.

First note the following fact about propositional logic: in general,
an implication of the form

\m{(z_1 \wedge z_2 \wedge \dots \wedge z_k) \rightarrow y}

is equivalent to the clause  \m{\{\overline{z_1},\dots,\overline{z_k},y\}}

\fh{Inside Group 6}

There are two subgroups here. The first just say that at any time
\m{i}, if cell \m{j} is not being scanned, then it will be \d{unchanged}
at time \m{i+1}. This is expressed by having the following clauses
for all \m{i,j,k} where \m{1 \le i < p(n)}, \m{-p(n) \le j \le p(n)}, \m{0 \le k \le v}:
\m{\{\overline{r_{i,j,k}},h_{i,j},r_{i+1,j,k}\}}

(In implicational form, this is 
\m{(r_{i,j,k} \wedge \overline{h_{i,j}}) \rightarrow r_{i+1,j,k}})

\fh{Inside Group 6}

The remaining subgroup in \m{G_6} depends on the transition function
\m{\delta}, e.g., suppose that \m{\delta(q_m,s_k) = 
\{(q_{m'}, s_{k'}, R)\}}.
Then we will have the following clauses for all \m{i}, \m{0 \le i \le p(n)}
and all \m{j}, \m{-p(n) \le j  \le p(n)}:

\m{\{\overline{y_{i,m}},\overline{h_{i,j}},\overline{r_{i,j,k}},y_{i+1,m'}\}}\\
\m{\{\overline{y_{i,m}},\overline{h_{i,j}},\overline{r_{i,j,k}},h_{i+1,j+1}\}}\\
\m{\{\overline{y_{i,m}},\overline{h_{i,j}},\overline{r_{i,j,k}},r_{i+1,j,k'}\}}

These again arise from implications, for example the first set from:

\m{(y_{i,m} \wedge h_{i,j} \wedge r_{i,j,k}) \rightarrow y_{i+1,m'}}

What happens if there's more than one choice for \m{\delta}?


\fh{More than one choice for transition}
 Suppose for \m{\delta(q_m,s_k)} there are \m{T} nondeterministic choices. For each possible value of \m{i} and \m{j}, add \m{T} variables   \m{z_{i,j,k,m,1},z_{i,j,k,m,2},\dots,z_{i,j,k,m,T}}

Now add a clause


\m{\{\overline{y_{i,m}},\overline{h_{i,j}},\overline{r_{i,j,k}},z_{i,j,k,m,1},z_{i,j,k,m,2},\dots,z_{i,j,k,m,T}\}}

which corresponds to the implication

\m{(y_{i,m} \wedge h_{i,j} \wedge r_{i,j,k}) \rightarrow (z_{i,j,k,m,1} \vee z_{i,j,k,m,2} \vee \dots \vee z_{i,j,k,m,T})}

Finally, for each possible value of \m{i,j,t} add clauses of the form
\m{\{\overline{z_{i,j,k,m,t}},y_{i+1,m'}\}},
\m{\{\overline{z_{i,j,k,m,t}},h_{i+1,j'}\}}, and
\m{\{\overline{z_{i,j,k,m,t}},r_{i+1,j,k'}\}}, where the exact values of \m{m',j'} and \m{k'} depend on the details of the \m{t}th alternative.

\fh{\m{f_L} is Polynomially Bounded}

It is not hard to see that:

\begin{itemize}
\item The number of clauses in each group is either constant (depends
only on \m{M}) or polynomial in \m{n = |x|}
\item The size of any clause is polynomial in \m{n = |x|} (note: all
clauses except the clause in \m{G_4} which specifies the input have
a constant number of clauses. Each variable can be encoded with
polynomial in \m{n} many bits.)
\end{itemize}

\fh{\m{f_L } is a reduction from \m{L} to SAT}

\begin{itemize}
\item
\m{w \in L} iff \m{f_L(w) \in SAT}.
\end{itemize} 
This is clear from the definition of the reduction.
\end{document}        
